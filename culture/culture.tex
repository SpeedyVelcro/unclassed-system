\chapter{Culture}\index{culture}\label{culture}

\begin{multicols*}{2}
    A character's culture is the group of people they grew up in. Cultures come
    with numerous bonuses, however, depending on the setting you're in, some
    cultures may be discriminated against or feared by those of other cultures.

    By default, you may choose one culture during character creation. To choose a
    culture, write it on your character sheet and apply all of the bonuses. Note
    that cultures sometimes grant skill modifiers, and sometimes points in a skill.
    Where a culture grants points in a skill, it cannot take you over that skill's
    potential.

    The cultures listed in this chapter are generic examples of cultures, and are
    not meant for normal play. Characters should ideally pick their cultures from
    a setting guide, or a list created by the GM. These cultures should usually
    encompass specific countries, tribes, or regions.

    \section{Multiple Cultures}
    Characters can have multiple cultures. This is typically the case if they spent
    some of their time before becoming an adventurer in one culture and then some
    of their time in another.

    Unlike mixing species, characters retain all the bonuses of all their cultures.
    Because having multiple cultures is usually more powerful, a character must
    spend 1 feat point for each culture they choose beyond the first.

    \import{./}{metropolitan}

    \import{./}{rural}

    \import{./}{tribal}

    \import{./}{martial}

    \import{./}{mountain-dwarf}

    \import{./}{high-elf}

    \import{./}{wood-elf}
\end{multicols*}
