\section{Designing Spells}

\begin{table*}[ht]
    \unclassedrowcolors
    \begin{tabularx}{\textwidth}{l X l X l X l X l X l X}
        \textbf{Dice} & \textbf{Cost} & \textbf{Dice} & \textbf{Cost} & \textbf{Dice} & \textbf{Cost} & \textbf{Dice} & \textbf{Cost} & \textbf{Dice} & \textbf{Cost} & \textbf{Dice} & \textbf{Cost} \\
        1d4 & 1   & 1d6 & 2    & 1d8 & 3    & 1d10 & 4    & 1d12 & 5    & 1d20 & 10    \\
        2d4 & 4   & 2d6 & 8    & 2d8 & 12   & 2d10 & 16   & 2d12 & 20   & 2d20 & 40    \\
        3d4 & 12  & 3d6 & 24   & 3d8 & 36   & 3d10 & 48   & 3d12 & 60   & 3d20 & 120   \\
        4d4 & 16  & 4d6 & 32   & 4d8 & 48   & 4d10 & 64   & 4d12 & 80   & 4d20 & 160   \\
        5d4 & 25  & 5d6 & 50   & 5d8 & 75   & 5d10 & 100  & 5d12 & 125  & 5d20 & 250   \\
        6d4 & 36  & 6d6 & 72   & 6d8 & 108  & 6d10 & 144  & 6d12 & 180  & 6d20 & 360   \\
        7d4 & 49  & 7d6 & 98   & 7d8 & 147  & 7d10 & 196  & 7d12 & 245  & 7d20 & 490   \\
        8d4 & 64  & 8d6 & 128  & 8d8 & 192  & 8d10 & 256  & 8d12 & 320  & 8d20 & 640   \\
        Xd4 & $\text{X}^\text{2}$ & Xd6 & $\text{2X}^\text{2}$ & Xd8 & $\text{3X}^\text{2}$ & Xd10 & $\text{4X}^\text{2}$ & Xd12 & $\text{5X}^\text{2}$ & Xd20 & $\text{10X}^\text{2}$ \\
    \end{tabularx}
    \caption{Spellcrafting Dice}
    \label{tab:spellcrafting-dice}
\end{table*}

This section contains advanced rules for designing spells. GMs may use these
rules as guidelines when creating their own spells, and players may craft
spells designed using this rules using the spellcrafting skill. (p.g.
\pageref{skill:spellcrafting}).

All spells in the spell list where created using these rules. The mana cost was
then adjusted (usually halved) to reflect that the most popular spells are the
ones crafted by powerful wizards to be as efficient as possible.

To design a minimal spell, you must choose at least one spell effect, specify
a target, and set the mana cost accordingly. There are additional decisions
you may make, which can effect the mana cost further.

\subsection{Spell Effects}
Spell effects are chosen from the spell effect list (see pg.
\pageref{spell-effect-list}). Some spell effects can be customised, in which
case instructions for doing so are usually within the spell effect description.

You may choose multiple spell effects for your spells. If you do, you add
together the mana costs, and then multiply the total by the number of spell
effects you have chosen (reflecting the added convenience of having multiple
effects on one spell). This multiplication comes after determining the mana
costs of individual spell effects, but before any other adjustments (such as
those based on the target).

You can determine the order in which the spell effects trigger.

\subsection{Spell School}
Each spell effect lists a spell school, which it confers on the spell itself.
Ideally, for easier casting, a spell should only use spell effects that share
a spell school.

If a spell uses spell effects of multiple different spells schools, its school
is all of those. If the spell calls for a skill check of the spell school of
the spell, the caster makes the skill check using their worst spell school out
of those.

\subsection{Target}
Broadly, there are two types of targeting: corporeal and incorporeal. Corporeal
targeting involves physically sending a spell to a target; for example, a icy
projectile, a bolt of lightning, or a fireball would all use corporeal
targeting.

The corporeal targets are as follows:
\begin{itemize}
    \item \textbf{Strike:} Choosing "strike" has no effect on mana cost. A
        strike spell applies its effects to a target in melee range if the
        caster succeeds on a skill check of the spell's school contested against
        the target's defence check. Spells with a strike target do not provoke
        opportunity attacks.
    \item \textbf{Missile:} "Missile" is chosen with a range in feet. For every
        5 ft. of range, a missile spell increases the mana cost by 1. A missile
        spell applies its effects to a target within the specified range if the
        caster succeeds on a skill check of the spell's school contested
        against the target's defence check.
        % TODO: Move all this malarkey to shared rules on area attacks in the
        % combat section, because most of it will also apply to grenade-type
        % weapons.
    \item \textbf{Explosion:} "Explosion" is chosen with a shape (cone, sphere, or
        cube) and the size of that shape. The mana cost of the spell is multiplied
        by the diameter (for spheres) or length of the shape in 5 ft. squares.
        The caster makes a skill check of the spell's school once. Targets
        within the area are then affected by the spell unless they can succeed
        on their contested defence skill check. Defenders cannot parry.
        Defenders cannot dodge unless they are standing on the edge of the
        shape and there is a space within 5 ft. to move to (they move there on
        a successful dodge). Defenders suffer half the spell's effect instead
        of bypassing it if they block.
    \item \textbf{Chain:} "Chain" is chosen with a range in feet, and a number
        of targets. Every 5 ft. of range increases the mana cost by 1, and the
        mana cost is then multiplied by the number of targets. The spell acts
        like a missile spell against its first target. Then it jumps to the
        closest creature or similarly conductive target - with ties broken
        randomly - and acts like a missile spell again. This repeats for the
        number of targets chosen.
\end{itemize}

Incorporeal targeting, on the other hand, governs spells with a non-physical
effect. You simply cast the spell and it automatically reaches its target (though
targets may be able to make rolls to resist the effect). This includes magic
such as mind-control. The incorporeal targets are: self, touch, ranged, or
area.

The incorporeal targets are as follows:
\begin{itemize}
    \item \textbf{Self:} Choosing "self" has no effect on mana cost. A self
        spell applies its effects to the caster.
    \item \textbf{Touch:} Choosing "touch" has no effect on mana cost. A touch
        spell applies its effects to a target in melee range.
    \item \textbf{Ranged:} "Ranged" is chosen with a range in feet. For every
        5 ft. of range, a ranged spell increases its mana cost by 1. A ranged
        spell applies its effects to a target within the specified range.
    \item \textbf{Area:} "Area" is chosen with a shape (cone, sphere, or cube)
        and the size of that shape. The mana cost of the spell is multiplied
        by the diameter (for spheres) or length of the shape in 5 ft. squares.
        The spell applies to all targets caught within the shape.
\end{itemize}

When you have multiple spell effects, you may only choose one target. You are
also not allowed to mix spells with corporeal and incorporeal targeting.

There is also a third (but less common) type of targeting: summoning. Summoning
targets a location in which to create an object, creature, or substance, and
the mana cost is calculated in the same way as ranged targeting.

\subsection{Duration}

\subsection{Casting Time}
% For those mathematically brave enough to craft spells that take longer than
% 24 hours to cast, it should be obvious enough that the final formula
% is intended to associate as (0.1 * (0.9^X))%, but brackets are omitted to
% make it look less scary
\begin{table*}[ht]
    \unclassedrowcolors
    \begin{tabularx}{\textwidth}{l X l X l X}
        \textbf{Casting Time} & \textbf{Mana Cost} & \textbf{Casting Time} & \textbf{Mana Cost} & \textbf{Casting Time} & \textbf{Mana Cost} \\
        1 round  & 90\% & 1 minute   & 10\% & 1 hour    & 1\% \\
        2 rounds & 80\% & 2 minutes  & 9\%  & 2 hours   & 0.9\%  \\
        3 rounds & 70\% & 3 minutes  & 8\%  & 3 hours   & 0.8\% \\
        4 rounds & 60\% & 5 minutes  & 7\%  & 4 hours   & 0.7\%  \\
        5 rounds & 50\% & 10 minutes & 6\%  & 6 hours   & 0.6\% \\
        6 rounds & 40\% & 15 minutes & 5\%  & 8 hours   & 0.5\% \\
        7 rounds & 30\% & 20 minutes & 4\%  & 16 hours  & 0.25\% \\
        8 rounds & 20\% & 30 minutes & 3\%  & 24 hours  & 0.1\% \\
                 &      & 45 minutes & 2\%  & 24X hours & 0.1 {\texttimes} 0.9\textsuperscript{X} \% \\
    \end{tabularx}
    \caption{Spellcrafting Casting Time}
    \label{tab:spellcrafting-casting-time}
\end{table*}
The default casting time of a spell is one action. This casting time
can be increased to any length of time listed in table
\ref{tab:spellcrafting-casting-time} in exchange for a reduced mana cost. The
mana cost is given as a percentage of the final mana cost.

The casting time of a spell can be reduced to a \textonehalf action. If so,
multiply the final mana cost by 4.

If a spell has spell effects with triggers, that spell can be cast
instantly - even out of turn - when the trigger condition is met. If a
spell is cast in response to a trigger like this, only spell effects
that have the same trigger are applied. This also means spell effects
with no trigger are skipped.

