
\begin{multicols*}{2}
    \section{Spellcrafting}
    \begin{table*}[ht]
        \unclassedrowcolors
        \begin{tabularx}{\textwidth}{l X l X l X l X l X l X}
            \textbf{Dice} & \textbf{Cost} & \textbf{Dice} & \textbf{Cost} & \textbf{Dice} & \textbf{Cost} & \textbf{Dice} & \textbf{Cost} & \textbf{Dice} & \textbf{Cost} & \textbf{Dice} & \textbf{Cost} \\
            1d4 & 1   & 1d6 & 2    & 1d8 & 3    & 1d10 & 4    & 1d12 & 5    & 1d20 & 10    \\
            2d4 & 4   & 2d6 & 8    & 2d8 & 12   & 2d10 & 16   & 2d12 & 20   & 2d20 & 40    \\
            3d4 & 12  & 3d6 & 24   & 3d8 & 36   & 3d10 & 48   & 3d12 & 60   & 3d20 & 120   \\
            4d4 & 16  & 4d6 & 32   & 4d8 & 48   & 4d10 & 64   & 4d12 & 80   & 4d20 & 160   \\
            5d4 & 25  & 5d6 & 50   & 5d8 & 75   & 5d10 & 100  & 5d12 & 125  & 5d20 & 250   \\
            6d4 & 36  & 6d6 & 72   & 6d8 & 108  & 6d10 & 144  & 6d12 & 180  & 6d20 & 360   \\
            7d4 & 49  & 7d6 & 98   & 7d8 & 147  & 7d10 & 196  & 7d12 & 245  & 7d20 & 490   \\
            8d4 & 64  & 8d6 & 128  & 8d8 & 192  & 8d10 & 256  & 8d12 & 320  & 8d20 & 640   \\
            Xd4 & $\text{X}^\text{2}$ & Xd6 & $\text{2X}^\text{2}$ & Xd8 & $\text{3X}^\text{2}$ & Xd10 & $\text{4X}^\text{2}$ & Xd12 & $\text{5X}^\text{2}$ & Xd20 & $\text{10X}^\text{2}$ \\
        \end{tabularx}
        \caption{Spellcrafting Dice}
        \label{tab:spellcrafting-dice}
    \end{table*}

    All spells in the spell list were originally created using spellcrafting.
    Characters can also create their own custom spells through spellcrafting.
    See the spellcrafting skill (pg. \pageref{skill:spellcrafting}) for
    instructions.

    Spells are constructed by specifying a list of spell effects, configuring
    those spell effects, and specifying a target. The mana cost is calculated
    based on these choices.

    \subsection{Spell Effects}
    Spell effects are chosen from the spell effect list (see pg.
    \pageref{spell-effect-list}). A spell can have any number of spell effects
    (though high numbers may make the spell difficult to cast or craft).

    To configure a spell, replace the values in square brackets. [number] should
    be replaced with a whole number, [duration] should be replaced with a number
    of rounds, and [dice] should be replaced with one of the dice from table
    \ref{tab:spellcrafting-dice}.


    \subsection{Target}
    Spell effects can be corporeal or incorporeal. Generally, if a target succeeds on a
    defence roll, no corporeal effects are applied to them. Incorporeal effects
    are applied regardless, however their descriptions often call for checks
    to determine whether they actually do anything.

    The target of a spell can be self, touch, ranged, location, area, or cone.
    You may only choose targets that are valid for all of a spell's spell
    effects.
    \begin{itemize}
        \item Spells with the "self" target affect the caster.
        \item Spells with the "touch" target affect an object or creature in
            touch range.
            % NB: Touch spells will not provoke attacks of opportunity. This
            % is their advantage over ranged spells.
        \item Spells with the "ranged" target affect any object or creature
            within a range specified by the spellcrafter.
            Corporeal effects only apply if the caster has line of effect
            to the target. Incorporeal effects only apply if the caster is
            aware of the specific location of the target (e.g. if they can see
            them).
        \item Spells with the "location" target affect a specific location
            within a range specified by the spell crafter. The caster must have
            line of effect to the target location.
        \item Spells with the "area" target affect all objects or creatures
            in an area chosen by the caster. The area's origin must be within
            a range specified by the spellcrafter. The area's shape is a cube
            or sphere, as specified by the spellcrafter. The area's size is
            specified by the spellcrafter.
        \item Spells with the "cone" target affect all objects or creatures
            caught within a cone originating from the caster. The length of
            the cone is specified by the spellcrafter.
    \end{itemize}

    Corporeal effects of area and cone spells cannot be parried. Dodging such
    spells only avoids their corporeal effects if the target is standing next
    to an unaffected space. Otherwise, dodging only reduces corporeal effects
    by half. Blocking such spells only reduces corporeal effects by half.

    \subsection{Mana Cost}
    To calculate the mana cost of a spell, first calculate the mana cost of
    each effect. A formula is given in the description of each spell effect.
    The total mana cost of the spell is then calculated using a formula given
    by the spell target, and rounded up:

    The mana cost of a spell can be calculated by following these steps:
    \begin{itemize}
        \item Calculate the mana cost of each spell effect as per its
            description.
        \item Sum the cost of all spell effects together.
        \item Multiply by the number of spell effects on the spell.
        \item If the target of the spell is area, multiply by the area's
            longest dimension divided by 5 ft.
        \item If the target of the spell is cone, multiply by the length of
            the cone.
        \item If the target of the spell is ranged, location, or area,
            add the range divided by 10 ft. and rounded up to the nearest
            whole number.
    \end{itemize}
    
    \section{Spell List}

    \section{Spell Effect List}
    \import{./spell-effects}{fire-damage}

\end{multicols*}{2}
