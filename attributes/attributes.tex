
\begin{table}[h!]
    \begin{tabular}{l l l}
        \multicolumn{3}{l}{\bfseries{Physical Skills}} \\
        \bfseries{ATT} & \bfseries{Attribute} & \bfseries{Description} \\
        STR & Strength & Raw physical strength; affects encumbrance and attack damage \\
        DEX & Dexterity & Agility and reflexes; affects combat order \\
        VIT & Vitality & Overall health and physical condition; affects hit points \\
        SEN & Senses & Sharpness and quality of physical senses; affects sight and hearing \\
        \multicolumn{3}{l}{\bfseries{Mental Skills}} \\
        \bfseries{ATT} & \bfseries{Attribute} & \bfseries{Description} \\
        INT & Intelligence & Logic, reasoning, and ability to learn; affects skill gain \\
        WIL & Willpower & Mental discipline and fortitude \\
        SOR & Sorcery & The ability to tap into and use magic \\
        ACU & Acumen & Practical wisdom and good judgement; "street-smarts" \\
        CHA & Charisma & Social graces \\
    \end{tabular}
    \caption{Attributes}
    \label{tab:attributes}
\end{table}

\section{Starting Attributes}
Generate a character's starting attributes using one of the following methods:
\begin{itemize}
    \item Refer to table \ref{tab:point-buy}, and choose values for all your
        attributes, summing up to a maximum of 60 points. This is the
        recommended method.
        % NB: The point buy for a peasant would be 24, as they have all 0s,
        % except sorcery, which would be -3.
    \item Set each attribute to -6, then for each attribute roll 4d4 and add
        the sum of the highest 3 dice. You may then swap any attributes you
        choose. This method skews towards more powerful characters than
        point buy, but comes with a risk of lower attributes.
    \item As above, but you are not allowed to swap attributes. This method
        makes every character a surprise.
\end{itemize}
Your GM may require you to use a specific method, so check before generating
your attributes.

\begin{table}[h!]
    \begin{tabular}{l l}
        \bfseries{Value} & \bfseries{Points} \\
        -3 & 0 \\
        -2 & 1 \\
        -1 & 2 \\
        0 & 3 \\
        1 & 5 \\
        2 & 8 \\
        3 & 12 \\
        4 & 18 \\
        5 & 30 \\
        6 & 54 \\
    \end{tabular}
    \caption{Point-buy}
    \label{tab:point-buy}
\end{table}
