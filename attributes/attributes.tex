\begin{multicols*}{2}
    \begin{table*}[ht]
        \unclassedrowcolors
        \begin{tabularx}{\textwidth}{l l X}
            \multicolumn{3}{l}{\textbf{Physical Attributes}} \\
            \textbf{ATT} & \textbf{Attribute} & \textbf{Description} \\
            STR & Strength & Raw physical strength; affects encumbrance and attack damage \\
            DEX & Dexterity & Agility and reflexes; affects combat order \\
            VIT & Vitality & Overall health and physical condition; affects hit points \\
            SEN & Senses & Sharpness and quality of physical senses; affects sight and hearing \\
            \multicolumn{3}{l}{\bfseries{Mental Attributes}} \\
            \textbf{ATT} & \textbf{Attribute} & \textbf{Description} \\
            INT & Intelligence & Logic, reasoning, and ability to learn; affects skill gain \\
            WIL & Willpower & Mental discipline and fortitude \\
            ACU & Acumen & Practical wisdom and good judgement; "street-smarts" \\
            CHA & Charisma & Social graces \\
            \multicolumn{3}{l}{\textbf{Special Attributes}} \\
            SOR & Sorcery & The ability to tap into and use magic \\
        \end{tabularx}
        \caption{Attributes}
        \label{tab:attributes}
    \end{table*}
    Attributes govern a character's general capabilities and affect nearly all
    aspects of gameplay. All skill checks include an attribute in their formulae,
    and some attributes have direct impacts on specific aspects of gameplay.

    While the Skills chapter (see pg. \ref{skills}) covers the effect of attributes
    on skill checks, this chapter outlines the direct impact of attributes on
    gameplay. A summary can be found in table \ref{tab:attributes}.

    The human average for any non-special attribute is 0.

    \section{Generation}
    By default, you may choose between two methods when generating attributes
    for a new character: point-buy and random. Point-buy allows you maximum
    control over your character and generally leads to more balanced
    attributes. Random has a chance of generating very powerful attributes, but
    at the risk of leaving you with much lower ones.
    
    \textbf{Point-Buy}: Pick attribute values from table \ref{tab:point-buy}
    for each attribute, up to a total point cost of 60.

    \textbf{Random}: Set each attribute to -10. For each attribute, roll 3d6
    and add the result. You may then swap attributes as you please.

    {
        \setlength\parindent{0pt}
        \unclassedrowcolors
        \begin{tabularx}{0.5\textwidth}{l X}
            \textbf{Attribute} & \textbf{Point Cost} \\
            -4 & 0 \\
            -3 & 1 \\
            -2 & 2 \\
            -1 & 3 \\
            0 & 4 \\
            1 & 5 \\
            2 & 7 \\
            3 & 10 \\
            4 & 15 \\
        \end{tabularx}
        \captionof{table}{Point-Buy}
        \label{tab:point-buy}
    }

    \subsection{Alternative Methods}
    The GM may decide to require the use of a specific method. In more balanced
    campaigns it is recommended to require all characters use point-buy. If a
    more chaotic or free-form campaign is desired, it is recommended all
    characters use random.

    The GM may also choose to allow or require  one of these alternative
    methods of attribute generation:
    
    \textbf{Balanced Random}: Set each attribute to -7. For each attribute,
    roll 3d4. You may then swap attributes as you please.

    \textbf{Powerful Random}: As random, but you roll 4d6 and drop the lowest.
    This method is skewed towards very powerful characters as there are many
    attributes so you are much more likely to get at least one extremely high
    attribute than in other systems with similar methods.

    \textbf{Fixed Random}: As random, but you may not swap attributes. This
    makes your character a surprise, and forces you to explore unexpected
    character builds.

    \textbf{Fixed Balanced Random}: As balanced random, but you may not swap
    attributes.

    \textbf{Fixed Powerful Random}: As powerful random, but you may not swap
    attributes.

\end{multicols*}



% NB: Commented out below is an earlier system for generating starting
% attributes
%\section{Starting Attributes}

%Generate a character's starting attributes using one of the following methods:
%\begin{itemize}
%    \item Refer to table \ref{tab:point-buy}, and choose values for all your
%        attributes, summing up to a maximum of 60 points. This is the
%        recommended method.
%        % NB: The point buy for a peasant would be 24, as they have all 0s,
%        % except sorcery, which would be -3.
%    \item Set each attribute to -6, then for each attribute roll 4d4 and add
%        the sum of the highest 3 dice. You may then swap any attributes you
%        choose. This method skews towards more powerful characters than
%        point buy, but comes with a risk of lower attributes.
%    \item As above, but you are not allowed to swap attributes. This method
%        makes every character a surprise.
%\end{itemize}
%Your GM may require you to use a specific method, so check before generating
%your attributes.
%
%\begin{table}[h!]
%    \begin{tabular}{l l}
%        \bfseries{Value} & \bfseries{Points} \\
%        -3 & 0 \\
%        -2 & 1 \\
%        -1 & 2 \\
%        0 & 3 \\
%        1 & 5 \\
%        2 & 8 \\
%        3 & 12 \\
%        4 & 18 \\
%        5 & 30 \\
%        6 & 54 \\
%    \end{tabular}
%    \caption{Point-buy}
%    \label{tab:point-buy}
%\end{table}

% NB: Another earlier method of stat-rolling
%All physical and mental attributes start at 0. Characters start without any
%special attributes, which can be marked with a hyphen (-) on your character
%sheet. In order to gain special attributes, you will usually need a specific
%feat.
%
%During character creation, you will raise and lower your attributes by 
%selecting a species (see pg. \ref{species}), culture (see pg. \ref{species}),
%background (see pg. \ref{background}), and feats (see pg. \ref{feats}).
%
%The effect of these choices on your attributes are detailed in their respective
%chapters.
