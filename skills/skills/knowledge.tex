\subsection{Knowledge}\index{knowledge}\label{skill:knowledge}

\noindent
\textbf{Governing Attribute:} INT

\noindent
\textbf{Default:} 0

\noindent
\textbf{Potential:} 4

\noindent
\textbf{Training Time:} 30 days

\noindent
\textbf{Self-Study:} Requires access to a library with at least the desired
knowledge (...) rating.

% TODO: Training (cost, time, extra stuff required)
% and factor in INT and allow self-teaching as an option

Knowledge affects a character's ability to recall facts and information.
Knowledge is divided into subtypes according to the subject of recall. The
subtypes are enumerated on table \ref{tab:knowledge-subtypes}.

Note that some knowledge skills are further divided by country. For example,
knowledge (history, Atlantis) would allow a character to recall historical
knowledge about Atlantis. The GM should tell players what countries they can
take such knowledge skills for. If the GM is not using a setting with
countries, such skills should not be divided by countries. In this case,
characters would be able to take knowledge (history) and recall any historical
knowledge.

\begin{center}
    \unclassedrowcolors
    \begin{tabularx}{0.5\textwidth}{X X}
        \textbf{Skill} & \textbf{Description} \\
        Knowledge (arcane) & Knowledge about magic peripheral to actual spellcasting \\
        Knowledge (astronomy) & Constellations and stellar systems if applicable to the setting \\
        Knowledge (history, [country]) & History involving a particular country \\
        Knowledge (nature) & Plants, animals, and ecosystems \\
        Knowledge (religion, [religion]) & Knowledge about the customs, history, myths, and traditions of a particular religion \\
    \end{tabularx}
    \captionof{table}{Knowledge subtypes}
    \label{tab:knowledge-subtypes}
\end{center}

When the GM calls for a knowledge skill check, the GM sets a target according
to the ubiquity of the knowledge based on the guidance below. Examples
for specific targets are given, but any arbitrary number can be picked.

A knowledge check of 0 recalls basic facts about reality, such as the country
a character is in, how many moons there are, or the fact that oceans exist.
Failing such a skill check will send others into abject disbelief and cause
them to question a character's sanity, possibly even damaging their dispostion.

A knowledge check of 5 recalls common knowledge that would be embarrassing to
forget, such as the name of the king, the price of bread, or identifying farm
animals.

A knowledge check of 10 recalls general knowledge that most people know, such
as recent wars, famous people, or religious customs. A knowledge check of 15
recalls similar general knowledge that a large minority of people know.

A knowledge check of 20 recalls obscure facts or knowledge obtained
from specialised study into a field. Note that a target of 20 becomes
accessible even to a character with an INT of 0 or a rating of 1 in knowledge, so
a target of 20 only represents facts obtained at the beginning of specialised
study. Esoteric secrets only become accessible at targets of 30 and beyond.

Knowledge checks of 40 usually represent the boundary of human knowledge.
Higher knowledge encroaches on the bounds of original research, with checks
of 80 or above representing earth-shattering secrets or even near-omnipotence.

Knowledge checks cannot be retried for a particular piece of information. If
a character still wants to find that information, they should consider using
the academia skill (pg. \pageref{academia}).

