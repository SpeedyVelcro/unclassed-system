\chapter{Basic Mechanics}\label{basics}
\begin{multicols*}{2}
    This chapter goes over the basic mechanics of the game, which are shared by all
    its different subsystems.
    
    \section{Dice Rolls}\label{dice}\index{dice}\index{die}
    Most mechanics in the Unclassed System are driven by the rolling of dice as a
    source of randomness.
    
    Dice rolls in this book are depicted by the letter "d" flanked by one or more
    numbers. The number after the "d" specifies the number of sides. For example, a
    six-sided dice that can generate the numbers 1-6 would be written as "d6". The
    number of this denomination of dice to roll is indicated by the number before
    the "d", so an instruction to roll 4 different 6-sided dice and add the results
    together would be written as "4d6". Dice rolls can be incorporated into
    mathematical formulae such as "1d4 + 2", which means roll a four-sided dice and
    add 2 to the result.
    
    The Unclassed System uses the traditional set of role-playing dice: a d4, a d6,
    a d8, a d10, a d12, and a d100. It is recommended that percentile dice (two
    d10-style dice that represent the units and tens column) are used to represent
    the d100.
    
    The Unclassed System also sometimes uses a d2 or d3. These can be represented
    using specialised dice, or alternatively using a coin and mapping a d6's sides
    to the numbers 1-3 respectively.

    \section{Bonuses and Penalties}\label{bonus}\label{penalty}\label{modifier}\index{bonus}\index{penalty}\index{modifier}
    Statistics in the Unclassed System can be modified by various modifiers, called
    bonuses when they are positive, and penalties when they are negative. For
    example, using a weapon in your off-hand might incur a -4 penalty to attack
    rolls, meaning you subtract 4 from the attack roll after making it, as well
    as applying any other bonuses or penalties.
    
    All modifiers come with one or more associated types. These are written as
    words preceeding the word bonus or penalty. Bonuses do not stack with bonuses
    of the same type, and likewise for penalties, but a bonus and a
    penalty of the same type \textit{do} stack, potentially cancelling each other
    out.
    
    When one or more modifiers apply that would not stack, only the modifier that
    is greatest in magnitude applies. The other modifiers still exist, but do not
    have any effect until those of greater magnitude expire.

    Note that due to the vast array of modifier types in the game, stacking bonuses
    and penalties should be possible more often than not. The design intent of the
    stacking rules is not to prevent character builds from stacking a large number
    of bonuses, but rather to prevent more extreme nonsensical cases from having
    an effect, such as drinking literally the same potion multiple times, or
    wearing multiple copies of the exact same ring.

    \subsection{Base Modifiers}\label{modifier}\index{modifier}
    Sometimes bonuses and penalties apply to a base statistic. This may be written,
    for example, as "+2 base SOR", which means a +2 bonus to base sorcery. These
    types of modifiers work differently to the non-base kind. A base modifier
    is applied as a one-shot, permanently changing the unmodified statistic, and
    can then be forgotten about as it is thereafter "baked in" to the statistic.

    Base modifiers do not have a type, and do not have any special rules preventing
    stacking, as they are ephemeral and only exist at the moment they are applied.

    Base modifiers are usually encountered in character creation when selecting
    species, cultures, and backgrounds.

    \section{Mathematics}
    Whenever mathematical formulae appear in this rulebook, standard
    mathematical notation is largely used, and the traditional order of
    operations applies. For readability though, variables are often written
    in uppercase (e.g. X, Y) in order to differentiate them from the lowercase
    "d", which is not a variable but rather dice roll notation (see pg.
    \pageref{dice}).

    \subsection{Rounding}
    Unless otherwise specified or included among the exceptions below, all
    values are rounded, rounding is always done to a whole number, and is
    always towards zero.

    Exceptions include weight and distance. Distance is rounded to a multiple
    of the grid space that is being used, usually 5 ft. Weight is rounded to
    a multiple of 0.1 lb.
\end{multicols*}{2}
