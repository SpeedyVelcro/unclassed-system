\chapter{Equipment}\index{equipment}\label{equipment}

Money can be exchanged for goods and services.

\section{Weapons}

\begin{multicols}{2}
    \section{Armor}

    \begin{table*}[ht!]
        \unclassedrowcolors
        \begin{tabularx}{\textwidth}{X l l l l}
            \unclassedsubtabletitle{5}{Suits of Armor} \\
            \textbf{Armor} & \textbf{Cost} & \textbf{Cover} & \textbf{Resistance} & \textbf{Weight} \\
            Hide & 5 sp & 4 & 1p/1s & 5 lb. \\
            Padded & 5 gp & 6 & 4b/2p/2s & 15 lb. \\
            Leather & 15 gp & 6 & 2b/1p/1s & 10 lb. \\
            % Following armors assume some sort of additional padding underneath, hence at least 4b
            Chain & 150 gp & 8 & 4b/2p/6s & 30 lb. \\ % Chain mail is highly resistant to slashing irl, weaker to piercing as you can force the rings apart
            Scale & 300 gp & 10 & 4b/4p/8s & 60 lb. \\ % Like chain mail, but improved resistance against thrusts and glances at certain angles
            Brigandine & 600 gp & 10 & 6b/8p/8s & 50 lb. \\
            Half Plate & 800 gp & 10 & 6b/16p/16s & 55 lb. \\ % Kinda like "munitions armor" - more gaps than full plate
            Full Plate & 1,500 gp & 14 & 8b/16p/16s/4m & 60 lb. \\ % Magical resistance only when wearing full plate in all slots
            \unclassedsubtabletitle{5}{Shields} \\
            \textbf{Armor} & \textbf{Cost} & \textbf{Cover} & \textbf{Resistance} & \textbf{Weight} \\
            Light steel shield & & & - & \\
        \end{tabularx}
        \caption{Armor}
        \label{tab:armor}
    \end{table*}

    The \textbf{cover rating} of a piece of armor represents the difficulty of
    finding a weak point and bypassing it. A higher cover rating means a greater \textit{chance}
    of reducing damage (see pg. \ref{combat} for how attacking works). The cover
    rating of a \textbf{shield} acts as a modifier to your block check (see pg.
    \ref{skill:block} for blocking).

    The \textbf{Armor Rating} of a piece of armor represents the difficulty of
    penetrating it. A higher armor rating means a greated \textit{damage reduction}
    when a piece of armor successfully covers you. Shields do not have an armor
    rating.

    The only penalty from armor is its weight. If you are not strong enough to
    carry your armor and all your equipment, you will incur potentially
    crippling encumbrance penalties. See the rules on encumbrance (pg.
    \ref{encumbrance}).

    Table \ref{tab:armor} shows the cost and weight of a full suit of armor. A full
    suit of armor is made of a helmet, a cuirass, greaves, a left boot, a right
    boot, a left gauntlet, a right gauntlet, a left pauldron and a right pauldron.
    All of these pieces have the same cover rating and armor rating as the full
    suit. Their cost and weight is a fixed proportion of that of a full suit, as
    shown in \ref{tab:armor-pieces}. Do not round.
    \begin{center}
        \unclassedrowcolors
        \begin{tabularx}{0.5\textwidth}{l l l}
            \textbf{Armor piece} & \textbf{Cost} & \textbf{Weight} \\
            Helmet & 0.1$\times$ & 0.1$\times$ \\
            Cuirass & 0.4$\times$ & 0.4$\times$ \\
            Greaves & 0.3$\times$ & 0.3$\times$ \\
            Boot & 0.05$\times$ & 0.05$\times$ \\
            Pauldron & 0.05$\times$ & 0.05$\times$ \\
            Gauntlet & 0.05$\times$ & 0.05$\times$ \\
        \end{tabularx}
        \captionof{table}{Armor Pieces}
        \label{tab:armor-pieces}
    \end{center}

    \section{Weapons}

    \section{Gear}

    \section{Goods}

    \section{Services}
\end{multicols}



\section{Goods}

\section{Services}
