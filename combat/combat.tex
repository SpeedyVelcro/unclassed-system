\chapter{Combat}\label{combat}

\begin{multicols*}{2}

    \section{Rounds, Turns, and Actions}\index{round}\index{turn}\index{action}

    \begin{table*}[ht]
        \unclassedrowcolors
        \begin{tabularx}{\textwidth}{X l l}
            \textbf{Activity} & \textbf{Cost} & \textbf{Provokes OA} \\
            Attack & 1 action & No \\
            Cast touch spell & Varies & No \\
            Cast ranged, cone, or area spell & Varies & Yes \\
            Cast self spell (or touch spell on self) & Varies & Yes \\
            Close door & \textonehalf\ action & No \\
            Draw weapon or take out item & \textonehalf\ action & No \\
            Drink potion & 1 action & Yes \\
            Drop weapon or item & Free action & No \\
            Dual-wield attack & 1 \textonehalf\ action & No \\
            Go prone & \textonehalf\ action & No \\
            Get up from prone & \textonehalf\ action & Yes \\
            Move half of speed & \textonehalf\ action & Varies \\
            Move full speed & 1 action & Varies \\
            Open door & \textonehalf\ action & Yes \\
            Sheathe weapon or stow item & \textonehalf\ action & Yes \\
            Sprint & Full turn & Varies \\
        \end{tabularx}
        \caption{Combat Actions}
        \label{tab:combat-actions}
    \end{table*}

    Combat in the Unclassed System is turn-based; each round, characters take
    turns in a specific order determined by initiative. Each character can
    take a default of 1 \textonehalf\ actions during their turn.

    Notice the extra \textonehalf\ action. Most characters will make 1 important
    action per turn - such as attacking or casting a spell - and then make a
    smaller \textonehalf\ action - such as moving half their speed or drawing a
    weapon. Actions can be done in any order, for example a character may first
    take a \textonehalf\ action to draw their sword and then take 1 action to
    attack. Any combination of actions can be taken up to a character's
    allottment of actions for that turn. For example, it is perfectly valid
    to take no actions or to take 3 separate \textonehalf\ actions in a turn.

    The length of various actions is listed on table \ref{tab:combat-actions}.

    One round represents 6 seconds of combat, however the turn system is an
    abstraction of this time. Although in game terms characters take turns to
    act, all the turns in a round are considered, in a narrative sense, to have
    some overlap. The order of turns is simply a representation of who's fast
    enough to make the first serious impact in combat.

    \section{Initiative}
    The order of turns in combat is determined by initiative. Each character
    involved in combat makes an initiative check at the start of combat by
    rolling 1d20 and adding their initiative modifier. By default, a
    character's initiative modifier is DEX + SEN, representing both the acrity
    of their senses and the speed at which their body reacts.

    Characters act in turn starting from the highest initiative and going to
    the lowest. Where two or more characters have the same initiative, ties
    are broken by the following heirarchy:
    \begin{itemize}
        \item Player characters and their allies go first
        \item Characters with the highest initiative modifier go first
        \item Characters who are still tied make another initiative check,
            following the tie-breaking steps again if necessary..
            This is not used as their actual initiative, but is just for
            comparison.
    \end{itemize}

    \section{Attacking}
    A character (the attacker) can use 1 action to attack another character
    (the defender). The attacker must have the appropriate weapon equipped
    for the weapon skill they are using, or a free hand if they are using
    the hand-to-hand skill. The process of attacking is as follows:
    \begin{itemize}
        \item Attacker makes an attack roll (weapon skill check)
        \item Defender makes a defence roll (dodge, block, or parry check)
        \item If the attack roll is less than or equal to the defense roll, the
                attack fails. Otherwise, subtract the defense roll from the
                attack roll.
        \item If the remainder of the attack roll is less than or equal to the
                defender's armor rating, the armor's damage reduction will be
                subtracted from the attack damage.
        \item Roll for damage
        \item Subtract damage reduction if applicable
        \item Apply damage
    \end{itemize}

    The attacker can use 1 \textonehalf\ actions to make a dual-wielding
    attack. A dual-wielding attack can only be made if the attacker has
    two weapons equipped. A free hand used with the hand-to-hand skill counts
    as a weapon for this purpose. To make a dual-wielding attack, follow the
    process above for making an attack once for each weapon.

    Attacks made with an off-hand take a -4 penalty to the weapon skill check.
    An off-hand is considered to be any hand other than a character's
    main-hand, as specified during character creation. Most characters'
    main-hand is their right hand.

    An attacker with an exotic configuration of hands, natural weapons,
    and/or double-ended weapons may be able to make a triple, quadruple,
    or even greater multi-wield attack. For such an attack, follow the rules
    for a dual-wielding attack and add an extra cost of a \textonehalf\ action
    for each extra attack that is made.

    \section{Moving}
    During combat, a character can use 1 action to move a distance up to their
    speed rounded down to a multiple of 5 ft. (but always at least 5 ft.).
    Alternatively, they can use a \textonehalf\ action to move a distance up
    to half their speed rounded in the same way.

    A character can also cover large distances in a turn by sprinting.
    Sprinting costs the entire turn, and can only be done if the character has
    not yet spent any of their action allottment. By default, a character can
    move 4 times their speed by sprinting. A character cannot move a distance
    greater than they can sprint in a turn by using multiple actions to move.

    Moving provokes opportunity attacks if and only if moving \textit{away}
    from other creatures. The opportunity attack is provoked only at the
    point where the moving character is moving from a space in melee range of
    the provoked character to a space further away.

    Moving from a space in melee range of another creature to a space further
    away from that creature provokes an opportunity attack. Only one such
    movement provokes an opportunity attack per other creature per turn.

    \section{Opportunity Attacks}
    Certain actions trigger opportunity attacks. When a character (the
    defender) makes such an action in melee range of another character (the
    attacker), the attacker may choose to make an opportunity attack. If they
    do, the attacker immediately makes an attack with a wielded melee weapon
    of choice against the defender. This attack takes place \textit{before}
    the defender carries out the action that triggered the opportunity attack.

    Certain combat skills can be used in place of an opportunity attack. These
    skills are grapple, trip, and disarm.

    By default each character can only make one opportunity attack per round.
\end{multicols*}
