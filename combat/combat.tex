\chapter{Combat}\label{combat}\index{combat}

\begin{multicols*}{2}

    \section{Passage of Time in Combat}
    While the passage of time in the Unclassed System is usually reasonably
    free-form and determined by the GM, stricter rules are applied to make
    sense of the flurry of activity that occurs during combat.

    Combat starts at the GMs discretion. Usually it should start as soon as
    any character attempts a hostile act towards another character. For
    example, the GM might declare that an NPC "reaches for their sword" and
    combat begins. They cannot actually draw their sword until their first
    turn in combat, but trying to draw their sword triggers combat.

    A player might attempt a hostile act towards another character, but it is
    still up to the GM to start combat. If for example, a player declares that
    they attack an NPC with their sword, the GM might narrate that "You begin
    to reach for your for your sword, and around you everyone springs into
    action", and combat begins. The player cannot actually draw their sword
    and attack until their turn (other characters may be able to react first
    if they have better initiative), and the GM should explain this to the
    player if they are unclear that they have not attacked yet.

    Combat also ends at the GMs discretion. Usually, this is when no immediate
    threat remains to any of the characters, such as when all of their foes
    have been killed.

    \subsection{Initiative}\index{initiative}
    The first thing all characters must do when entering combat is to make an
    initiative skill check. Initiative represents a character's reaction time,
    and how quickly they can spring into action and make the first blow. As
    initiative is such an iconic precursor to combat across RPGs, the GM will
    often signal combat has started by simply saying "everyone roll for
    initiative".

    The order of turns in combat is determined by each character's initiative
    skill check. Characters take turns starting from the highest initiative
    and going to the lowest. Where two or more characters have the same
    initiative, ties are broken by the following heirarchy:
    \begin{itemize}
        \item Player characters and their allies go first
        \item Characters with the highest initiative modifier go first
        \item Characters who are still tied make another initiative check,
            following the tie-breaking steps again if necessary..
            This is not used as their actual initiative, but is just for
            comparison.
    \end{itemize}

    \subsection{Rounds and Turns}\index{round}\index{turn}

    \begin{table*}[ht]
        \unclassedrowcolors
        \begin{tabularx}{\textwidth}{X l l}
            \textbf{Activity} & \textbf{Cost} & \textbf{Provokes OA} \\
            Attack & 1 action & No \\
            Cast touch spell & Varies & No \\
            Cast ranged, cone, or area spell & Varies & Yes \\
            Cast self spell (or touch spell on self) & Varies & Yes \\
            Close door & \textonehalf\ action & No \\
            Draw weapon or take out item & \textonehalf\ action & No \\
            Drink potion & 1 action & Yes \\
            Drop weapon or item & Free action & No \\
            Dual-wield attack & 1 \textonehalf\ action & No \\
            Go prone & \textonehalf\ action & No \\
            Get up from prone & \textonehalf\ action & Yes \\
            Move half of speed & \textonehalf\ action & Varies \\
            Move full speed & 1 action & Varies \\
            Open door & \textonehalf\ action & Yes \\
            Sheathe weapon or stow item & \textonehalf\ action & Yes \\
            Sprint & Full turn & Varies \\
        \end{tabularx}
        \caption{Combat Actions}
        \label{tab:combat-actions}
    \end{table*}

    Combat in the Unclassed System is turn-based; each round, characters take
    turns in a specific order determined by initiative, and they can perform
    a specific number of actions each turn.

    One round represents 6 seconds of combat; the turn system is an
    abstraction of this time. Although, in game terms, characters take turns to
    act, all the turns in a round are considered, in a narrative sense, to
    happen at roughly the same time. The order of turns is simply a
    representation of who reacts faster.

    \subsection{Actions}\index{action}
    Each character can take a default of 1 \textonehalf\ actions during
    their turn.

    Notice the extra \textonehalf\ action. Most characters will make 1 important
    action per turn - such as attacking or casting a spell - and then make a
    smaller \textonehalf\ action - such as moving half their speed or drawing a
    weapon. The length of various activities is listed on table
    \ref{tab:combat-actions}.

    Actions can be done in any order, for example a character may first
    take a \textonehalf\ action to draw their sword and then take 1 action to
    attack. A character can also combine any number of actions and
    \textonehalf\ actions up to their allotment for that turn. A character
    could, for example, use three \textonehalf\ actions in a turn instead of
    an action and a \textonehalf\ action. An example of this might be stowing
    away a crossbow, drawing a sword, and moving half their speed towards an
    opponent.

    Some activities are called "free actions"\index{free action}. These are
    activities that you can carry out on your turn, but do not require any
    actions to use.

    \subsection{Reactions}\index{reaction}
    Reactions are a special kind of action that you can use out of your turn
    under specific circumstances. The most common reaction seen in combat is
    the opportunity attack (pg. \pageref{combat:opportunity-attacks}). A
    reaction details a specific trigger, and characters can only use a reaction
    when the trigger condition is fulfilled.

    A character can, by default, use only one reaction per round, measured
    from the start of their turn to the start of their turn next round.

    Some reactions are called "free reactions"\index{free reaction}. These
    reactions can be used by characters as many times per round as they want,
    without using up their allotment of actions per round.

    \section{Attacking}\label{combat:attacking}
    A character (the attacker) can use 1 action to attack another character
    (the defender). The attacker must have the appropriate weapon equipped
    for the weapon skill they are using, or a free hand if they are using
    the hand-to-hand skill. The process of attacking is as follows:
    \begin{itemize}
        \item Attacker makes an attack check (weapon skill check)
        \item Defender makes a defence check (dodge, block, or parry check)
        \item If the attack check is greater than or equal to the defence
            check, it hits.
        \item If the attack check is greater than or equal to the defender's
            cover rating\index{cover rating} (granted by their armor), then
            the attack also bypasses armor\index{armor}.
        \item If the attack hit, roll for damage.
        \item Subtract any resistances\index{resistance} from the damage based
            on the weapon's damage type. Only subtract resistance from armor
            if the attack did \textit{not} bypass armor.
        \item Apply the damage, if any. Attacks cannot deal negative
            damage.
    \end{itemize}

    Attacks made with a one-hand weapon being wielded in an off-hand take a -4
    penalty to the weapon skill check. An off-hand is considered to be any hand
    other than a character's main-hand, as specified during character creation.
    Most characters' main-hand is their right hand. An unarmed attack with an
    off-hand fist does not have such a penalty.

    A two-handed weapon does not have an off-hand penalty unless you choose to
    equip it backwards for some reason.

    \subsection{Dual-wielding}\label{combat:dual-wielding}
    A character can use 1 \textonehalf\ actions to make a dual-wielding
    attack. A dual-wielding attack is made with two separate weapons - or a
    double-ended weapons - that are currently equipped. This can include
    natural weapons that aren't being used to wield a weapon. In a
    dual-wielding attack, a character simply makes a separate attack with each
    weapon.

    An attacker with three or more equipped weapons may be able to make a
    triple, quadruple, or even greater multi-wield attack. For such an attack,
    follow the rules for a dual-wielding attack and add an extra cost of a
    \textonehalf\ action for each extra attack that is made.

    \subsection{Exotic wields}
    Two-handed weapons can be wielded one-handed. A two-handed melee weapon
    wielded one-handed has a -4 penalty to the attack roll in a character's
    main hand, and a -6 penalty in their off-hand. Two-handed ranged weapons
    cannot be wielded one-handed with only one hand or body part.

    Weapons can be wielded with body parts other than hands. If the body part
    can conceivably "grip" the weapon if enough force were applied (e.g. teeth,
    feet, or two body parts in combination such as elbows or knees) apply a -5
    penalty multiplied by the weapon's weight in lb. and rounded to a whole
    number in the negative direction. For two-handed weapons, double this
    penalty unless one of the body parts is a hand.

    \section{Moving}\label{combat:moving}
    During combat, a character can use 1 action to move a distance up to their
    speed rounded down to a multiple of 5 ft. (but always at least 5 ft.).
    Alternatively, they can use a \textonehalf\ action to move a distance up
    to half their speed rounded in the same way.

    A character can also cover large distances in a turn by sprinting.
    Sprinting costs the entire turn, and can only be done if the character has
    not yet spent any of their action allottment. By default, a character can
    move 4 times their speed by sprinting. A character cannot move a distance
    greater than they can sprint in a turn by using multiple actions to move.


    Moving from a space in melee range of another creature to a space further
    away from that creature provokes an opportunity attack. Only one such
    movement provokes an opportunity attack per other creature per turn.

    \section{Opportunity Attacks}\label{combat:opportunity-attacks}\index{opportunity-attack}
    Certain activities create an opening that opponents can capitalise on to
    make make an attack. These activities trigger opportunity attacks, and
    are marked on table \ref{tab:combat-actions}.

    When a character attempts an activity that triggers an opportunity attack,
    other characters within range are first given the chance to use a reaction
    to perform an opportunity attack with a melee weapon they are wielding.
    Opportunity attacks are otherwise exactly the same as a regular attack.

    Moving provokes opportunity attacks if and only if moving \textit{away}
    from other creatures. The opportunity attack is provoked only at the
    point where the moving character is moving from a space in melee range of
    the provoked character to a space further away. If the opportunity attack
    results in the moving character becoming prone or grappled, that character
    immediately stops moving.

    A character cannot make multiple opportunity attacks against the same
    character per round (measured from the start of their turn to the start
    of their next turn), but can make as many opportunity attacks as they have
    reactions against multiple different characters.

    Certain combat skills can be used in place of an opportunity attack. These
    skills are grapple, trip, and disarm.

    \section{Critical Hits}\label{combat:critical-hits}\index{critical hit}
    Critical hits represent especially successful attacks: those where luck and
    skill align to deal a devastating blow to an opponent e.g. striking a weak
    spot or finding a chink in armor.

    To make a critical hit, a character must first score a critical
    threat\index{critical threat} on an attack. An attack scores a critical
    threat if the natural roll (i.e. the number displayed on the d20 itself
    without modifiers) is within the weapon's critical range. Most weapons
    have a critical range of 20, in which case only a natural 20 scores a
    critical threat. Other critical ranges are listed in table
    \ref{tab:weapons} on page \pageref{tab:weapons}.

    On a critical threat, attacks automatically hit. If they would have hit
    anyway, they also bypass armor.

    Immediately after scoring a critical threat, and before rolling damage, a
    character makes a second attack check, but the target keeps their existing
    check. If the new attack check is a hit against the defence check, the
    critical threat is upgraded to a critical hit.

    A critical hit is an automatic hit, bypasses armor, and deals increased
    damage. Once a character confirms a critical hit, they roll for damage
    as normal, but multiply the result (of the entire damage formula including
    modifiers) by the weapon's critical damage multiplier. Multipliers are
    listed in table \ref{tab:weapons} on page \pageref{tab:weapons}.
\end{multicols*}
