\chapter{Combat}\label{combat}\index{combat}

\begin{multicols*}{2}

    \section{Passage of Time in Combat}
    While the passage of time in the Unclassed System is usually reasonably
    free-form and determined by the GM, stricter rules are applied to make
    sense of the flurry of activity that occurs during combat.

    Combat starts at the GMs discretion. Usually it should start as soon as
    any character attempts a hostile act towards another character. For
    example, the GM might declare that an NPC "reaches for their sword" and
    combat begins. They cannot actually draw their sword until their first
    turn in combat, but trying to draw their sword triggers combat.

    A player might attempt a hostile act towards another character, but it is
    still up to the GM to start combat. If for example, a player declares that
    they attack an NPC with their sword, the GM might narrate that "You begin
    to reach for your for your sword, and around you everyone springs into
    action", and combat begins. The player cannot actually draw their sword
    and attack until their turn (other characters may be able to react first
    if they have better initiative), and the GM should explain this to the
    player if they are unclear that they have not attacked yet.

    Combat also ends at the GMs discretion. Usually, this is when no immediate
    threat remains to any of the characters, such as when all of their foes
    have been killed.

    \subsection{Initiative}\index{initiative}
    The first thing all characters must do when entering combat is to make an
    initiative skill check. Initiative represents a character's reaction time,
    and how quickly they can spring into action and make the first blow. As
    initiative is such an iconic precursor to combat across RPGs, the GM will
    often signal combat has started by simply saying "everyone roll for
    initiative".

    The order of turns in combat is determined by each character's initiative
    skill check. Characters take turns starting from the highest initiative
    and going to the lowest. Where two or more characters have the same
    initiative, ties are broken by the following heirarchy:
    \begin{itemize}
        \item Player characters and their allies go first
        \item Characters with the highest initiative modifier go first
        \item Characters who are still tied make another initiative check,
            following the tie-breaking steps again if necessary..
            This is not used as their actual initiative, but is just for
            comparison.
    \end{itemize}

    \subsection{Rounds and Turns}\index{round}\index{turn}

    \begin{table*}[ht]
        \unclassedrowcolors
        \begin{tabularx}{\textwidth}{X l l}
            \textbf{Activity} & \textbf{Cost} & \textbf{Provokes OA} \\
            Attack & 1 action & No \\
            Cast touch spell & Varies & No \\
            Cast ranged, cone, or area spell & Varies & Yes \\
            Cast self spell (or touch spell on self) & Varies & Yes \\
            Close door & \textonehalf\ action & No \\
            Draw weapon or take out item & \textonehalf\ action & No \\
            Drink potion & \textonehalf\ action & Yes \\
            Drop weapon or item & Free action & No \\
            Go prone & \textonehalf\ action & No \\
            Get up from prone & \textonehalf\ action & Yes \\
            Open door & \textonehalf\ action & Yes \\
            Sheathe weapon or stow item & \textonehalf\ action & Yes \\
            Sprint & Full turn & Varies \\
        \end{tabularx}
        \caption{Combat Actions}
        \label{tab:combat-actions}
    \end{table*}

    Combat in the Unclassed System is turn-based; each round, characters take
    turns in a specific order determined by initiative, and they can perform
    a specific number of actions each turn.

    One round represents 6 seconds of combat; the turn system is an
    abstraction of this time. Although, in game terms, characters take turns to
    act, all the turns in a round are considered, in a narrative sense, to
    happen at roughly the same time. The order of turns is simply a
    representation of who reacts faster.

    \subsection{Actions}\index{action}
    By default, a character can use a body part to perform 1 action per turn.
    For example, a character with a "fighter" style of build might use their
    right hand's action to attack with a sword, while leaving their left hand
    unused because they are holding a shield. A wizard might use their right
    hand to cast a spell, while leaving their left hand unused because they
    don't need an extra hand in that moment. Meanwhile, a character with a
    "berserker" style build focused around dual-wielding might perform an
    attack with their right hand and an attack with their left hand.

    A character's allotment of actions is tracked per body part, enabling
    multitasking. A duelist running low on health, for example, might attack
    with a rapier held in their right hand, while using an
    action to take a potion out of their pack and quaff it. Usually,
    performing an action with another hand applies a penalty to attacks (see
    the rules for attacking, pg. \pageref{combat:attacking}) so the duelist
    would find it harder to hit their opponent while splitting their attention
    between their potion and their weapon.

    Moving is not an action, but notably interacts any action a character takes
    using their feet (see pg. \pageref{combat:movement}). A rogue making a
    daring escape might attack a pursuer with a dagger held in their right
    hand, while opening a door with their left hand. In this case, the rogue
    has left their feet unused so that they can be used for movement, however
    if the feet decides to move, they may incur a penalty on their attack due
    to a lack of sure footing. Meanwhile, a monk who kicks an opponent will
    find their movement that turn severely limited, as they have spent most of
    their turn with one foot in the air.

    Some actions are less significant than attacking, and leave a character
    free to perform another small action. This is the action
    cost\index{action cost} and is typically listed as a fraction. A character
    can take 2 separate \textonehalf\ actions whenever they could perform 1
    action, for example. Action costs are listed in table
    \ref{tab:combat-actions}.

    Some actions have no cost. These are called free
    actions\index{free action}, and can be performed as many times as a
    character wants during a turn between any other actions. Performing a free
    action does not count as having performed an action for mechanics that
    would be affected. For example, a character performing a free action with
    their left hand would not incur a penalty on an attack made with their
    right hand.

    \subsection{Reactions}\index{reaction}
    Reactions are a special kind of action that you can use out of your turn
    under specific circumstances. The most common reaction seen in combat is
    the opportunity attack (pg. \pageref{combat:opportunity-attacks}). A
    reaction details a specific trigger, and characters can only use a reaction
    when the trigger condition is fulfilled.

    A character can, by default, use only one reaction per round, measured
    from the start of their turn to the start of their turn next round.

    Some reactions are called "free reactions"\index{free reaction}. These
    reactions can be used by characters as many times per round as they want,
    without using up their allotment of reactions per round.

    \section{Attacking}\label{combat:attacking}
    A character (the attacker) can use 1 action to attack another character
    (the defender). The attacker must have the appropriate weapon equipped
    for the weapon skill they are using, or a free hand (or other body part)
    if they are using the hand-to-hand skill. The process of attacking is as
    follows:
    \begin{itemize}
        \item Attacker makes an attack check (weapon skill check)
        \item Defender makes a defence check (dodge, block, or parry check)
        \item If the attack check is greater than or equal to the defence
            check, it hits.
        \item If the attack check is greater than or equal to the defence check
            plus the defender's cover rating\index{cover rating} (granted by
            their armor), then the attack also bypasses armor\index{armor}.
        \item If the attack hit, roll for damage.
        \item Subtract any resistances\index{resistance} from the damage based
            on the weapon's damage type. Only subtract resistance from armor
            if the attack did \textit{not} bypass armor.
        \item Apply the damage, if any. Attacks cannot deal negative damage.
    \end{itemize}

    Attacks made with a one-hand weapon being wielded in an off-hand take a -4
    handedness penalty to the weapon skill check. An off-hand is considered to
    be any hand other than a character's main-hand, as specified during
    character creation. Most characters' main-hand is their right hand.
    An unarmed attack with an off-hand fist does not have such a penalty.

    A two-handed weapon does not have an off-hand penalty unless a character
    chooses to equip it backwards for some reason, or if a creature with an
    unusual configuration of hands uses only off-hands to wield it.

    \subsection{Multitasking}\label{combat:multitasking}\index{multitasking}
    When attacking, the attacker takes a -4 multitasking penalty for each other
    action performed simultaneously using other body parts (scaling with the
    action cost, so -2 for each \textonehalf\ action). For example, a character
    trying to dual-wield\index{dual-wielding} weapons might make an attack
    with a sword wielded in their right hand, at the same time as making an
    attack with a dagger wielded in their left hand. The attack with their
    sword would take a -4 multitasking penalty and, assuming they are
    right-handed, the attack with their dagger would take a -4 multitasking
    penalty \textit{and} a -4 handedness penalty.

    If an attacker can take 2 or more actions per turn per body part (usually
    because they have taken a feat), they can avoid multitasking penalties by
    timing their actions. Multitasking penalties only apply if they overlap.
    For example, a rogue trying to make an escape may be able to make two
    actions per turn with each body part. With their right hand, they attack
    a pursuer with their dagger. With their left hand, they want to open a
    door, but doing so would incur a multitasking penalty. Instead, they skip
    1 action with their left hand, and then perform a \textonehalf\ action to
    open the door after the attack action has completed. This incurs no
    multitasking penalty on the attack.

    \subsection{Exotic wields}
    Two-handed weapons can be wielded one-handed. A two-handed melee weapon
    wielded one-handed has a -4 handedness penalty to the attack roll in a
    character's main hand, and a -6 handedness penalty in their off-hand.
    Two-handed ranged weapons cannot be wielded one-handed with only one hand
    or body part.

    Weapons can be wielded with body parts other than hands. If the body part
    can conceivably "grip" the weapon if enough force were applied (e.g. teeth,
    feet, or two body parts in combination such as elbows or knees) apply a -3
    handedness penalty multiplied by the weapon's weight in lb. and rounded to
    a whole number in the negative direction. For two-handed weapons, double
    this penalty unless one of the body parts is a hand.

    \section{Moving}\label{combat:moving}
    During combat, a character can move a distance up to their speed each turn
    as long as they do not take any actions using their feet that turn.

    A character that takes any actions using their feet is limited to moving
    5 ft. in a turn.

    A character can cover larger distances in a turn by sprinting. Usually,
    this allows a character to move up to twice their speed in a turn.
    Sprinting requires a character to engage their entire body, so can only be
    done if no actions are performed using any body part in a turn.

    Moving from a space in melee range of another creature to a space further
    away from that creature provokes an opportunity attack (pg.
    \pageref{combat:opportunity-attacks}). Only one such movement provokes an
    opportunity attack from each creature in a turn, regardless of the number
    of reactions that creature has.

    \subsection{Exotic Movement}
    Characters with exotic configurations of legs can move up to half their
    speed in a turn if some of their feet are engaged in actions, but at least
    2 are available. If all - or all but one - of their feet are engaged in actions,
    they can move up to 5 ft. as usual.

    \section{Opportunity Attacks}\label{combat:opportunity-attacks}\index{opportunity attack}
    Certain activities create an opening that opponents can capitalise on to
    make make an attack. These activities trigger opportunity attacks, and
    are marked on table \ref{tab:combat-actions}.

    When a character attempts an activity that triggers an opportunity attack,
    other characters within range are first given the chance to use a reaction
    to perform an opportunity attack with a melee weapon they are wielding.
    Opportunity attacks are otherwise exactly the same as a regular attack.

    Moving provokes opportunity attacks if and only if moving \textit{away}
    from other creatures, and moving more than 5 ft. in a turn. The opportunity
    attack is provoked only at the point where the moving character is moving
    from a space in melee range of the provoked character to a space further
    away. If the opportunity attack
    results in the moving character becoming prone or grappled, that character
    immediately stops moving.

    A character cannot make multiple opportunity attacks against the same
    character per round (measured from the start of their turn to the start
    of their next turn), but can make as many opportunity attacks as they have
    reactions against multiple different characters.

    Certain combat skills can be used in place of an opportunity attack. These
    skills are grapple, trip, and disarm.

    \section{Critical Hits}\label{combat:critical-hits}\index{critical hit}
    Critical hits represent especially successful attacks: those where luck and
    skill align to deal a devastating blow to an opponent e.g. striking a weak
    spot or finding a chink in armor.

    To make a critical hit, a character must first score a critical
    threat\index{critical threat} on an attack. An attack scores a critical
    threat if the natural roll (i.e. the number displayed on the d20 itself
    without modifiers) is within the weapon's critical threat
    range\index{critical threat range}. Critical threat ranges are listed in
    table \ref{tab:weapons} on page \pageref{tab:weapons}.

    On a critical threat, attacks automatically hit, but they don't necessarily
    bypass armor.

    Immediately after scoring a critical threat, and before rolling damage, the
    attacker confirms the critical hit\index{critical confirmation} by rolling
    another d20. If the natural roll falls within the attacker's critical
    confirmation range\index{critical confirmation range} (only a roll
    of 20 by default), then the critical threat is upgraded to a critical hit.

    A critical hit is an automatic hit, bypasses armor, and deals increased
    damage. Once the attacker confirms a critical hit, they roll for damage
    as normal, but multiply the result (of the entire damage formula including
    modifiers) by the weapon's critical damage
    multiplier\index{critical damage multiplier}. Multipliers are
    listed in table \ref{tab:weapons} on page \pageref{tab:weapons}.
\end{multicols*}
