\chapter{Combat}\label{combat}

\begin{multicols*}{2}

    \section{Rounds, Turns, and Actions}\index{round}\index{turn}\index{action}
    Combat in the Unclassed System is turn-based; each round, characters take
    turns in a specific order determined by initiative. Each character can
    take a default of 1 \textonehalf actions during their turn.

    Notice the extra \textonehalf action. Most characters will make 1 important
    action per turn - such as attacking or casting a spell - and then make a
    smaller \textonehalf action - such as moving half their speed or drawing a
    weapon. Actions can be done in any order, for example a character may first
    take a \textonehalf action to draw their sword and then take 1 action to
    attack. Any combination of actions can be taken up to a character's
    allottment of actions for that turn. For example, it is perfectly valid
    to take no actions or to take 3 separate \textonehalf actions in a turn.

    One round represents 6 seconds of combat, however the turn system is an
    abstraction of this time. Although in game terms characters take turns to
    act, all the turns in a round are considered, in a narrative sense, to have
    some overlap. The order of turns is simply a representation of who's fast
    enough to make the first serious impact in combat.

    \section{Initiative}
    The order of turns in combat is determined by initiative. Each character
    involved in combat makes an initiative check at the start of combat by
    rolling 1d20 and adding their initiative modifier. By default, a
    character's initiative modifier is DEX + SEN, representing both the acrity
    of their senses and the speed at which their body reacts.

    Characters act in turn starting from the highest initiative and going to
    the lowest. Where two or more characters have the same initiative, ties
    are broken by the following heirarchy:
    \begin{itemize}
        \item Player characters and their allies go first
        \item Characters with the highest initiative modifier go first
        \item Characters who are still tied make a second initiative check.
            This is not used as their actual initiative, but is just for
            comparison.
    \end{itemize}

    \section{Attacking}
    A character (the attacker) can use 1 action to attack another character
    (the defender). The attacker must have the appropriate weapon equipped
    for the weapon skill they are using, or a free hand if they are using
    the hand-to-hand skill. The process of attacking is as follows:
    \begin{itemize}
        \item Attacker makes an attack roll (weapon skill check)
        \item Defender makes a defence roll (dodge, block, or parry check)
        \item If the attack roll is less than or equal to the defense roll, the
                attack fails. Otherwise, subtract the defense roll from the
                attack roll.
        \item If the remainder of the attack roll is less than or equal to the
                defender's armor rating, the armor's damage reduction will be
                subtracted from the attack damage.
        \item Roll for damage
        \item Subtract damage reduction if applicable
        \item Apply damage
    \end{itemize}

    The attacker can use 1 \textonehalf actions to make a dual-wielding
    attack. A dual-wielding attack can only be made if the attacker has
    two weapons equipped. A free hand used with the hand-to-hand skill counts
    as a weapon for this purpose. To make a dual-wielding attack, follow the
    process above for making an attack once for each weapon.

    Attacks made with an off-hand take a -4 penalty to the weapon skill check.
    An off-hand is considered to be any hand other than a character's
    main-hand, as specified during character creation. Most characters'
    main-hand is their right hand.

    An attacker with an exotic configuration of hands, natural weapons,
    and/or double-ended weapons may be able to make a triple, quadruple,
    or even greater multi-wield attack. For such an attack, follow the rules
    for a dual-wielding attack and add an extra cost of a \textonehalf action
    for each extra attack that is made.
\end{multicols*}
